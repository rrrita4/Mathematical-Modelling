\documentclass{article}
\textheight 23.5cm \textwidth 15.8cm
%\leftskip -1cm
\topmargin -1.5cm \oddsidemargin 0.3cm \evensidemargin -0.3cm
%\documentclass[final]{siamltex}

\usepackage{graphicx}
\usepackage{epsfig}
\usepackage{amsmath}
\usepackage{amsthm}
\usepackage{amssymb}
\usepackage{mathrsfs}
\usepackage{float}
\usepackage{multirow}
\usepackage{verbatim}
\usepackage{fancyhdr}
\usepackage{subfigure}
\usepackage{color}
\usepackage{mathtools}
\usepackage{mathrsfs}
%\usepackage{natbib}
\usepackage{sectsty}
%\usepackage[title]{appendix}
\usepackage{threeparttable}
\usepackage{dcolumn}
\usepackage{booktabs}
\usepackage{indentfirst}
\usepackage{setspace}
\usepackage{bm}
\usepackage{enumerate}
\usepackage{geometry}
\usepackage{fontspec}
\setmainfont[Mapping=tex-text]{KaiTi}

\title{社科研讨课——人口增长预测问题}
\author{杨隽祎,黄小科,郑伟一}

\begin{document}
\maketitle

\section{Introduction}

近年来人口增长成为一大热点,作为人口大国,对人口的预测影响着我国政策的制定与实施,进行恰当的建
模与数据处理,在此切实应用的问题中,熟练对知识的掌握与应用。

\section{Question}

中国是一个人口大国,人口问题始终是制约我国发展的关键因素之一。根据已有数据,运用数学建模的方法,
对中国人口做出分析和预测。

从中国的实际情况和人口增长的上述特点出发,参考相关数据,建立中国人口增长的数学模型,并由此对中
国人口增长的中短期和长期趋势做出预测,特别要指出模型中的优点与不足之处。


\section{Learning And Discussion}

本题来源于实际问题,需要真实有效的数据。确定好方向后,我们三人便从网络寻找数据,例如近些年来人
口总数,出生率,死亡率,各年龄段人口总数等。

此次我们做了稍微调整,讨论学习相关数据处理,学习讨论模型的合理性,多角度切入,在结果处给出模拟
结果与真实情况的对比,以此评定模型的合理性,进而分析相关结论。

\section{Difficulty}

尽管有上学期运筹学建模题目的引导,我们组仍然面对诸多问题。

1.  模型建立较为多样,如何选择合适的模型。

2.  较为专业的内容难以涉猎。

3.  自身知识的有所欠缺,编程能力较为薄弱

4.  疫情影响,前期准备阶段组员之间线下联系受阻。

\section{Solution}

认识到上述问题,我们组讨论后决定,改变上学期分组方式,采取三人讨论给出若干相关模型,再由组员分
别切入不同模型,同时编程处理数据,最终讨论各个模型的优劣。

其次我们数据均来自国家统计局(《中国统计年鉴》)。模型知识学习自姜启源老师的《数学模型》及其译
制的《数学建模》。



\section{Model one}

\subsection{model}
模型:灰色动态模型-GM(1,1)

将输入的数据记作一列数据($记为:x_0=(x_0(1),...,x_0(n))$),引入累加,逆累加,均值,级比生成算子:

累加生成:将原序列的数据依次累加的到生成序列($记为:x_1=(x_1(1),...,x_1(n))$)

逆累加生成:将生成序列相邻数据作差生成新序列($记为:y_1=(y_1(1),...,y_1(n))$)

均值生成:生成序列相邻数据作均值生成新序列($记为:z_1=(z_1(2),...,z_1(n))$)

级比生成:原序列相邻数据作比生成新序列($记为:\sigma=(\sigma(2),...,\sigma(n))$)

数学表达:

$x_1(k)=\sum_{m=1}^kx_0(m)$\qquad\qquad\qquad\qquad\;$y_1(k)=x_1(k)-x_1(k-1)$

$z_1(k)=\frac{(x_1(k)+x_1(k-1))}{2}$\qquad\qquad\qquad\qquad$\sigma(k)=\frac{x_0(k-1)}{x_0(k)}$

则定义序列间基本关系:$x_0(k)+az_1(k)=b$

其中设
\begin{equation}
	P=
	\begin{pmatrix}
		a\\
		b
	\end{pmatrix}
\quad
	Y=
	\begin{pmatrix}
		x_0(2)\\
		x_0(3)\\
		...\\
		x_0(n)
	\end{pmatrix}
\quad
	B=
	\begin{pmatrix}
		-z_1(2) & 1\\
		-z_1(3) & 1\\
		... & ...\\
		-z_1(n) & 1
	\end{pmatrix}
\end{equation}

最小二乘法估计参数列:$\hat{P}=(\hat{a},\hat{b})^T=(B^TB)^{-1}B^TY$

利用离散数据序列建立近似的微分方程模型:$\frac{dx_1}{dt}+ax_1=b$

解得时间相应函数:$x_1(t)=(x_0(1)-\frac{b}{a})e^{-ak}+\frac{b}{a}$

由此可得原始数据的预测值:
$$
\hat{x}_0(k+1)=\hat{x}_1(k+1)-\hat{x}_1(k)
=(1-e^{\hat{a}})(x_0{1}-\frac{\hat{b}}{\hat{a}})e^{-\hat{a}k}
$$

精度检验引入残差,相对误差的定义:

残差:$q(k)=x_0(k)-\hat{x}_0(k)$

相对误差:$\epsilon(k)=\frac{q(k)}{x_0(k)}\times100\%$

精度:$p^0=(1-\frac{1}{n-1}\sum_{k=2}^n\left|\epsilon(k)\right|)\times100\%$

\subsection{application}


\section{Model Two}

\section{Conclusion}

Summarize your findings and add your comments here.


\section{Reflection}

从薄利多销问题到人口预测,两道题目均是运筹学范畴。无论是建模还是对数据的处理,误差分析,都要求
我们对曲线的拟合讨论,这恰恰是二者最重要的关系。相比薄利多销问题,人口预测问题对模型处理,编程
能力都提出了更高的要求。其中的Leslie模型更是能将现有的线代知识加以应用,深化认知。

\end{document}